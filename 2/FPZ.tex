\documentclass{article}
\usepackage [english, russian] { babel }

\begin{document}
\textbf{Неформальная постановка задачи}

Дано N независимых работ, для каждой работы задано время выполнения. Требуется построить расписание выполнения работ без прерываний на M процессорах. На расписании должно достигаться минимальное значение разбалансированности расписания (т.е. значения разности Tmax-Tmin, где Tmax - наибольшая, по всем процессорам, длительность расписания на процессоре; Tmin - аналогично, наименьшая длительность) (критерий К3).

\textbf{Формальная постановка задачи}

\textit{Дано}:
\begin{itemize}
\item Множество работ $P = \{p_i\}$, где $p_i = \{N_i, W_i\}$, где $N_i$ -- номер работы, $W_i$ -- продолжительность работы.
\item Множество процессоров $M = \{m_i\}$.
\end{itemize}

Определим расписание $HP$ как пару $\{HP_B, HP_L\}$, где $HP_B: P \rightarrow M$ (каждой работе сопоставляется процессор, на котором она будет выполняться), а $HP_L = \{p_{i_j}\}$ -- упорядоченное множество, задающее порядок выполнения работ.

\textit{Требуется}:
\begin{itemize}
\item Построить расписание $HP$.
\end{itemize}

Определим множество $T$ как $\{\sum\limits_{p_i: HP_B(p_i) = m_i} W_i | m_i \in M\}$

\textit{Минимизируемый критерий}:
\begin{itemize}
\item $max(T) - min(T)$.
\end{itemize}
\end{document}
